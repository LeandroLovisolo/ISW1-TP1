\documentclass[a4paper, 10pt, twoside]{article}

\usepackage[top=1in, bottom=1in, left=1in, right=1in]{geometry}
\usepackage[utf8]{inputenc}
\usepackage[spanish, es-ucroman, es-noquoting]{babel}
\usepackage{setspace}
\usepackage{fancyhdr}
\usepackage{lastpage}
\usepackage{amsmath}
\usepackage{amsfonts}
\usepackage{amsthm}
\usepackage{verbatim}
\usepackage{graphicx}
\usepackage{float}
\usepackage{enumitem} % Provee macro \setlist
\usepackage{tabularx}
\usepackage{multirow}
\usepackage{hyperref}
\usepackage{multicol}
\usepackage[toc, page]{appendix}


%%%%%%%%%% Configuración de Fancyhdr - Inicio %%%%%%%%%%
\pagestyle{fancy}
\thispagestyle{fancy}
\lhead{Trabajo Práctico 1 · Ingeniería de Software I}
\rhead{Delgado · Lovisolo · Petaccio · Requeni · Vita}
\renewcommand{\footrulewidth}{0.4pt}
\cfoot{\thepage /\pageref{LastPage}}

\fancypagestyle{caratula} {
   \fancyhf{}
   \cfoot{\thepage /\pageref{LastPage}}
   \renewcommand{\headrulewidth}{0pt}
   \renewcommand{\footrulewidth}{0pt}
}
%%%%%%%%%% Configuración de Fancyhdr - Fin %%%%%%%%%%


%%%%%%%%%% Miscelánea - Inicio %%%%%%%%%%
% Evita que el documento se estire verticalmente para ocupar el espacio vacío
% en cada página.
\raggedbottom

% Deshabilita sangría en la primer línea de un párrafo.
\setlength{\parindent}{0em}

% Separación entre párrafos.
\setlength{\parskip}{0.5em}

% Separación entre elementos de listas.
\setlist{itemsep=0.5em}

% Asigna la traducción de la palabra 'Appendices'.
\renewcommand{\appendixtocname}{Apéndices}
\renewcommand{\appendixpagename}{Apéndices}
%%%%%%%%%% Miscelánea - Fin %%%%%%%%%%


%%%%%%%%%% Insertar diagrama - Inicio %%%%%%%%%%
\newcommand{\diagrama}[1]{
  \includegraphics[width=16cm]{#1}
}
%%%%%%%%%% Insertar diagrama - Fin %%%%%%%%%%


\begin{document}


%%%%%%%%%%%%%%%%%%%%%%%%%%%%%%%%%%%%%%%%%%%%%%%%%%%%%%%%%%%%%%%%%%%%%%%%%%%%%%%
%% Carátula                                                                  %%
%%%%%%%%%%%%%%%%%%%%%%%%%%%%%%%%%%%%%%%%%%%%%%%%%%%%%%%%%%%%%%%%%%%%%%%%%%%%%%%


\thispagestyle{caratula}

\begin{center}

\includegraphics[height=2cm]{DC.png} 
\hfill
\includegraphics[height=2cm]{UBA.jpg} 

\vspace{2cm}

Departamento de Computación,\\
Facultad de Ciencias Exactas y Naturales,\\
Universidad de Buenos Aires

\vspace{4cm}

\begin{Huge}
Trabajo Práctico 1
\end{Huge}

\vspace{0.5cm}

\begin{Large}
Ingeniería de Software I
\end{Large}

\vspace{1cm}

Primer Cuatrimestre de 2014

\vspace{4cm}

\begin{tabular}{|c|c|c|}
\hline
Apellido y Nombre & LU & E-mail\\
\hline
Delgado Alejandro N.  & 601/11 & nahueldelgado@gmail.com\\
Lovisolo Leandro      & 645/11 & leandro@leandro.me\\
Petaccio Lautaro José & 443/11 & lausuper@gmail.com\\
Requeni Gastón        & 400/11 & grequeni@hotmail.com\\
Vita Sebastián        & 149/11 & sebastian\_vita@yahoo.com.ar\\
\hline
\end{tabular}

\end{center}

\newpage


%%%%%%%%%%%%%%%%%%%%%%%%%%%%%%%%%%%%%%%%%%%%%%%%%%%%%%%%%%%%%%%%%%%%%%%%%%%%%%%
%% Índice                                                                    %%
%%%%%%%%%%%%%%%%%%%%%%%%%%%%%%%%%%%%%%%%%%%%%%%%%%%%%%%%%%%%%%%%%%%%%%%%%%%%%%%


\tableofcontents

\newpage


%%%%%%%%%%%%%%%%%%%%%%%%%%%%%%%%%%%%%%%%%%%%%%%%%%%%%%%%%%%%%%%%%%%%%%%%%%%%%%%
%% Introducción                                                              %%
%%%%%%%%%%%%%%%%%%%%%%%%%%%%%%%%%%%%%%%%%%%%%%%%%%%%%%%%%%%%%%%%%%%%%%%%%%%%%%%


\section{Introducción}

En este trabajo práctico aplicamos varias técnicas de ingeniería de requerimientos para modelar un hipotético sistema de administración de la red de ciclovías de una ciudad. Además, proporcionamos una lista de escenarios informales que ejemplifican situaciones representativas del funcionamiento esperado.


%%%%%%%%%%%%%%%%%%%%%%%%%%%%%%%%%%%%%%%%%%%%%%%%%%%%%%%%%%%%%%%%%%%%%%%%%%%%%%%
%% Modelo de Jackson                                                         %%
%%%%%%%%%%%%%%%%%%%%%%%%%%%%%%%%%%%%%%%%%%%%%%%%%%%%%%%%%%%%%%%%%%%%%%%%%%%%%%%


\section{Modelo de Jackson}


\subsection{Mundo}

\begin{multicols}{2}
  \begin{itemize}
    \item Usuario retira bicicleta	
    \item Usuario usa la bicicleta más de una hora
    \item Usuario no consigue bicicleta
    \item Usuario sufre el robo de una bicicleta
    \item Usuario reporta al personal de la estación que su bicicleta fue robada	
    \item Usuario tiene accidente en bicicleta
    \item Usuarios salen de trabajar en una franja horaria reducida	
    \item Usuario espera bicicleta
    \item Usuario espera bicicleta por más de 40 minutos 
    \item Usuario roba identidad a otro
    \item Usuario penalizado retira bicicleta	
    \item Usuario sin registrar retira bicicleta
    \item Usuario retorna bicicleta tarde (más de 1hr) y se lo penaliza
    \item Usuario va a buscar una bici a estación del centro
    \item Usuario va a buscar una bici a estación periférica
    \item Usuario retorna bicicleta en horario 
    \item Personal de empresa de transporte trasladan bicicletas
    \item Personal de empresa de transporte tardan en trasladar bicicletas
    \item Personal de empresa de transporte lleva bicicletas a una estación
    \item Personal de empresa de transporte retira bicicletas de una estación
    \item Empresa de transporte recibe solicitud de transporte de bicis de estación A a B 
    \item Sistema pierde conexión
    \item Estacion pierde conexión
    \item Estación se queda sin bicicletas
    \item Personal de estación entrega una bicicleta 
    \item Personal de estación recibe una bicicleta
    \item Personal de estación rechaza una petición de retirar bicicleta debido suspensión por infracciones
    \item Personal de estación rechaza una petición de retirar bicicleta ya que el usuario ya tiene una bicicleta
    \item Personal de estación rechaza una petición de retirar bicicleta debido a que el usuario no es válido
    \item Personal de estación rechaza una petición de retirar una bicicleta ya que no hay stock
    \item Usuario se identifica con el personal de una estación
    \item Se cae la conexión de la estación
    \item Personal del gobierno de Mar Chiquita entrega nuevas bicicletas al personal de la empresa de transporte
  \end{itemize}
\end{multicols}


\subsection{Interfaz}

\begin{multicols}{2}
  \begin{itemize}
    \item Usuario se registra en el sistema
    \item Usuario consulta disponibilidad de bicicletas
    \item Usuario consulta su estado de penalización
    \item Personal de estación informa al sistema que una bicicleta fue robada
    \item Personal de estación informa al sistema el retiro de una bicicleta por parte de un usuario determinado
    \item Personal de estación informa al sistema la devolución de una bicicleta por parte de un usuario determinado
    \item Personal de estación solicita reposición de bicicletas
    \item Personal de estación autentifica un usuario en el sistema
    \item Personal de estación consulta con el sistema el retiro de una bicicleta por parte de un usuario determinado
    \item Sistema informa al personal de la estación si la petición de retirar bicicleta de un usuario es aceptada/rechazada
    \item Sistema informa al personal de la estación si la petición de autenticación de un usuario resulta correcta/incorrecta
    \item Sistema informa al usuario el resultado del registro
    \item Sistema informa al usuario el estado de penalización de este
    \item Sistema informa al usuario la disponibilidad de bicicletas
    \item Sistema envía ordenes de reposición de bicicletas al personal de la empresa de transporte
    \item Personal del gobierno de Mar Chiquita se autentica en el sistema
    \item Personal del gobierno de Mar Chiquita informa al sistema la disponibilidad de nuevas bicicletas
    \item Personal del gobierno de Mar Chiquita informa al sistema que existe una nueva estación

  \end{itemize}
\end{multicols}


\subsection{Sistema}

\begin{multicols}{2}
  \begin{itemize}
    \item Sistema checkea que un usuario registrado no vuelva a registrarse
    \item Sistema checkea si existen penalizaciones
    \item Sistema checkea si ya se sacó una bici
    \item Sistema penaliza usuarios por pérdia/robo de bicicleta
    \item Sistema calcula la penalización de los usuarios infractores basado en el tiempo excedido del límite de 1h
    \item Sistema registra usuario
    \item Sistema registra una nueva estacion
    \item Sistema registra nuevas bicicletas agregadas por el personal del estado
    \item Sistema penaliza usuario por pérdida de bicicleta
    \item Sistema utiliza un algoritmo de encriptación seguro para las contraseñas de los usuarios
    \item Sistema lleva registro de cantidad de bicis en cada estación 
    \item Sistema lleva registro de todos los usuarios existentes y si tienen o no una bici en su poder
    \item Sistema diferencia entre estaciones del centro y periféricas 
    \item Sistema trata de que las estaciones del centro tengan más bicis que las periféricas a todo momento (si se desbalancea mucho, repone automáticamente)
    \item Sistema le indica al personal de empresa de transporte cuantas bicicletas deben quitar de una estación y a donde entregarlas.
    \item Sistema calcula mediante un algoritmo que estación es la mas indicada para ceder bicicletas según la distancia entre las estaciones (?)
  \end{itemize}
\end{multicols}


%%%%%%%%%%%%%%%%%%%%%%%%%%%%%%%%%%%%%%%%%%%%%%%%%%%%%%%%%%%%%%%%%%%%%%%%%%%%%%%
%% Diagrama de Contexto                                                      %%
%%%%%%%%%%%%%%%%%%%%%%%%%%%%%%%%%%%%%%%%%%%%%%%%%%%%%%%%%%%%%%%%%%%%%%%%%%%%%%%


\section{Diagrama de Contexto}

\diagrama{diagrama-de-contexto.png}


%%%%%%%%%%%%%%%%%%%%%%%%%%%%%%%%%%%%%%%%%%%%%%%%%%%%%%%%%%%%%%%%%%%%%%%%%%%%%%%
%% Modelo de Objetivos                                                       %%
%%%%%%%%%%%%%%%%%%%%%%%%%%%%%%%%%%%%%%%%%%%%%%%%%%%%%%%%%%%%%%%%%%%%%%%%%%%%%%%


\section{Modelo de Objetivos}

\diagrama{images/LN_6.png}


%%%%%%%%%%%%%%%%%%%%%%%%%%%%%%%%%%%%%%%%%%%%%%%%%%%%%%%%%%%%%%%%%%%%%%%%%%%%%%%
%% Escenarios hipotéticos                                                    %%
%%%%%%%%%%%%%%%%%%%%%%%%%%%%%%%%%%%%%%%%%%%%%%%%%%%%%%%%%%%%%%%%%%%%%%%%%%%%%%%


\section{Escenarios Hipotéticos}


Los siguientes escenarios ilustran de manera informal situaciones representativas del funcionamiento esperado del sistema.


\subsection{Escenario 1: Registro de usuario}

Un habitante de Mar Chiquita desea comenzar a utilizar las bicicletas del sistema como su medio de transporte. Para conseguirlo ingresa a la página web en donde se encuentra el formulario de registro de usuarios, y procede a completarlo con su nombre completo, su número de DNI, su dirección de correo electrónico y una contraseña para proteger el acceso a sus datos de uso del sistema. Al envíar los datos solicitados, se le confirma que la registración fue exitosa. A partir de ese momento, este habitante ya es considerado un usuario del servicio.


\subsection{Escenario 2: Usuario retira bicicleta, y más tarde la devuelve}

Un usuario registrado se dirige a una estación de la red de ciclovías y expresa su intención de retirar una bicicleta al empleado de la estación. Éste le solicita el DNI al usuario con el fin de verificar su identidad con el mismo y también para poder cargar correctamente los datos del retiro de la bicicleta en el sistema. Luego de la verificación, el empleado devuelve el DNI, y si el usuario no se encuentra penalizado, permite al usuario a tomar una bicicleta. A continuación ingresa la solicitud en el sistema utilizando el DNI del usuario y el ID de la bicicleta; el sistema registra el retiro con fecha y hora del momento.

Luego el usuario ya puede comenzar a utilizar la bicicleta retirada. Se traslada hasta la próxima estación en su recorrido en menos de una hora, y expresa allí su intención de devolver la bicicleta. El empleado de esta otra estación le solicita el DNI al usuario para verificar su identidad y poder registrar la devolución en el sistema. Luego de la verificación, carga la devolución en la interfaz del software indicando número de DNI del usuario, ID de la bicicleta y estado de la misma (sana/rota) al momento de la devolución. El sistema registra la devolución con fecha y hora, y el usuario se retira.


\subsection{Escenario 3: Usuario intenta realizar el retiro/devolución en una estación offline}

Un usuario se dirige a una estación para poder retirar/devolver una bicicleta, pero la estación se encuentra sin servicio de internet. Luego de realizar la verificación de la identidad del usuario mediante su DNI, al momento de registrar la operación en el sistema de software, en lugar de eso el empleado de la estación se comunica por handy con otra estación que sí posee conectividad a internet, y delega el registro de la operación a su compañero de la segunda estación. Si el usuario pretendía devolver una bicicleta, la misma es aceptada por el empleado de la estación. Y en el caso de que haya querido retirar una bicicleta, si hay stock, se le ofrece una de las disponibles en la estación.


\subsection{Escenario 3: Usuario devuelve bicicleta después de una hora de retirarla}

Un usuario se dirige a una estación de bicicletas para devolver la bicicleta que tiene en su poder, pero transcurrió más de una hora desde que la retiró (el tiempo máximo permitido de retención de una bicicleta).

El proceso de devolución transcurre por su curso normal, pero una vez que el empleado de la estación carga la devolución en la interfaz del software, el sistema marca al usuario como penalizado y le envía un mail notificándole el hecho. Mientras el usuario esté penalizado, no podrá retirar bicicletas. Asímismo se le ofrece pagar una multa para normalizar su situación.

Luego de recibir la bicicleta del usuario, éste se retira de la estación.


\end{document}