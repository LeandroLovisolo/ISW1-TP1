\documentclass[a4paper, 10pt, twoside]{article}

\usepackage[top=1in, bottom=1in, left=1in, right=1in]{geometry}
\usepackage[utf8]{inputenc}
\usepackage[spanish, es-ucroman, es-noquoting]{babel}
\usepackage{setspace}
\usepackage{fancyhdr}
\usepackage{lastpage}
\usepackage{amsmath}
\usepackage{amsfonts}
\usepackage{amsthm}
\usepackage{verbatim}
\usepackage{graphicx}
\usepackage{float}
\usepackage{enumitem} % Provee macro \setlist
\usepackage{tabularx}
\usepackage{multirow}
\usepackage{hyperref}
\usepackage{multicol}
\usepackage[toc, page]{appendix}


%%%%%%%%%% Configuración de Fancyhdr - Inicio %%%%%%%%%%
\pagestyle{fancy}
\thispagestyle{fancy}
\lhead{Trabajo Práctico 1 · Ingeniería de Software I}
\rhead{Delgado · Lovisolo · Petaccio · Requeni · Vita}
\renewcommand{\footrulewidth}{0.4pt}
\cfoot{\thepage /\pageref{LastPage}}

\fancypagestyle{caratula} {
   \fancyhf{}
   \cfoot{\thepage /\pageref{LastPage}}
   \renewcommand{\headrulewidth}{0pt}
   \renewcommand{\footrulewidth}{0pt}
}
%%%%%%%%%% Configuración de Fancyhdr - Fin %%%%%%%%%%


%%%%%%%%%% Miscelánea - Inicio %%%%%%%%%%
% Evita que el documento se estire verticalmente para ocupar el espacio vacío
% en cada página.
\raggedbottom

% Deshabilita sangría en la primer línea de un párrafo.
\setlength{\parindent}{0em}

% Separación entre párrafos.
\setlength{\parskip}{0.5em}

% Separación entre elementos de listas.
\setlist{itemsep=0.5em}

% Asigna la traducción de la palabra 'Appendices'.
\renewcommand{\appendixtocname}{Apéndices}
\renewcommand{\appendixpagename}{Apéndices}
%%%%%%%%%% Miscelánea - Fin %%%%%%%%%%


%%%%%%%%%% Insertar diagrama - Inicio %%%%%%%%%%
\newcommand{\diagrama}[1]{
  \includegraphics[type=png,ext=.png,read=.png,width=16cm]{#1}
}
%%%%%%%%%% Insertar diagrama - Fin %%%%%%%%%%

\begin{document}


%%%%%%%%%%%%%%%%%%%%%%%%%%%%%%%%%%%%%%%%%%%%%%%%%%%%%%%%%%%%%%%%%%%%%%%%%%%%%%%
%% Carátula                                                                  %%
%%%%%%%%%%%%%%%%%%%%%%%%%%%%%%%%%%%%%%%%%%%%%%%%%%%%%%%%%%%%%%%%%%%%%%%%%%%%%%%


\thispagestyle{caratula}

\begin{center}

\includegraphics[height=2cm]{DC.png} 
\hfill
\includegraphics[height=2cm]{UBA.jpg} 

\vspace{2cm}

Departamento de Computación,\\
Facultad de Ciencias Exactas y Naturales,\\
Universidad de Buenos Aires

\vspace{4cm}

\begin{Huge}
Trabajo Práctico 1
\end{Huge}

\vspace{0.5cm}

\begin{Large}
Ingeniería de Software I
\end{Large}

\vspace{1cm}

Primer Cuatrimestre de 2014

\vspace{4cm}

\begin{tabular}{|c|c|c|}
\hline
Apellido y Nombre & LU & E-mail\\
\hline
Delgado, Alejandro N.  & 601/11 & nahueldelgado@gmail.com\\
Lovisolo, Leandro      & 645/11 & leandro@leandro.me\\
Petaccio, Lautaro José & 443/11 & lausuper@gmail.com\\
Requeni, Gastón        & 400/11 & grequeni@hotmail.com\\
Vita, Sebastián        & 149/11 & sebastian\_vita@yahoo.com.ar\\
\hline
\end{tabular}

\end{center}

\newpage


%%%%%%%%%%%%%%%%%%%%%%%%%%%%%%%%%%%%%%%%%%%%%%%%%%%%%%%%%%%%%%%%%%%%%%%%%%%%%%%
%% Índice                                                                    %%
%%%%%%%%%%%%%%%%%%%%%%%%%%%%%%%%%%%%%%%%%%%%%%%%%%%%%%%%%%%%%%%%%%%%%%%%%%%%%%%


\tableofcontents

\newpage


%%%%%%%%%%%%%%%%%%%%%%%%%%%%%%%%%%%%%%%%%%%%%%%%%%%%%%%%%%%%%%%%%%%%%%%%%%%%%%%
%% Introducción                                                              %%
%%%%%%%%%%%%%%%%%%%%%%%%%%%%%%%%%%%%%%%%%%%%%%%%%%%%%%%%%%%%%%%%%%%%%%%%%%%%%%%


\section{Introducción}

En este trabajo práctico aplicamos varias técnicas de ingeniería de requerimientos para modelar un hipotético sistema de administración de la red de ciclovías de una ciudad. Además, proporcionamos una lista de escenarios informales que ejemplifican situaciones representativas del funcionamiento esperado.


%%%%%%%%%%%%%%%%%%%%%%%%%%%%%%%%%%%%%%%%%%%%%%%%%%%%%%%%%%%%%%%%%%%%%%%%%%%%%%%
%% Diagrama de Contexto                                                      %%
%%%%%%%%%%%%%%%%%%%%%%%%%%%%%%%%%%%%%%%%%%%%%%%%%%%%%%%%%%%%%%%%%%%%%%%%%%%%%%%


\section{Diagrama de contexto}

\diagrama{diagrama-de-contexto}

A continuación desarrollamos algunas de las interacciones más importantes.


\subsection{Nombre de la interacción 1}

Pendiente.


\subsection{Nombre de la interacción 2}

Pendiente.


\subsection{Nombre de la interacción 3}

Pendiente.


%%%%%%%%%%%%%%%%%%%%%%%%%%%%%%%%%%%%%%%%%%%%%%%%%%%%%%%%%%%%%%%%%%%%%%%%%%%%%%%
%% Modelo de Objetivos                                                       %%
%%%%%%%%%%%%%%%%%%%%%%%%%%%%%%%%%%%%%%%%%%%%%%%%%%%%%%%%%%%%%%%%%%%%%%%%%%%%%%%


\section{Modelo de objetivos}

\subsection{(1.1)  Objetivos principales}
\diagrama{objetivos-1.1}

\subsection{(1.2)  Objetivos principales}
\diagrama{objetivos-1.2}

\subsection{(1.3)  Objetivos principales}
\diagrama{objetivos-1.3}

\subsection{(1.4)  Objetivos principales}
\diagrama{objetivos-1.4}

\subsection{(2)    Registro de usuarios por internet}
\diagrama{objetivos-2}

\subsection{(3)    Consulta de disponibilidad por internet}
\diagrama{objetivos-3}

\subsection{(4)    Consulta penalización por internet}
\diagrama{objetivos-4}

\subsection{(5.1)  Solicitud de bicicletas: hay stock, usuario habilitado}
\diagrama{objetivos-5.1}

\subsection{(5.2)  Solicitud de bicicletas: hay stock, usuario habilitado}
\diagrama{objetivos-5.2}

\subsection{(5.3)  Solicitud de bicicletas: hay stock, usuario habilitado}
\diagrama{objetivos-5.3}

\subsection{(6)    Solicitud de bicicletas: hay stock, usuario no habilitado}
\diagrama{objetivos-6}

\subsection{(7)    Solicitud de bicicletas: no hay stock}
\diagrama{objetivos-7}

\subsection{(8)    Devolución de bicicletas}
\diagrama{objetivos-8}

\subsection{(9)    Evitar robos de identidad}
\diagrama{objetivos-9}

\subsection{(10)   Penalizaciones}
\diagrama{objetivos-10}

\subsection{(11)   Penalizaciones: retención por más de una hora}
\diagrama{objetivos-11}

\subsection{(12)   Penalizaciones: bicicleta dañada}
\diagrama{objetivos-12}

\subsection{(13)   Penalizaciones: denuncia de robo}
\diagrama{objetivos-13}

\subsection{(14)   Penalizaciones: intercambio de bicicletas}
\diagrama{objetivos-14}

\subsection{(15)   Penalizaciones: pago de multas}
\diagrama{objetivos-15}

\subsection{(16)   Reabastecimiento: apertura}
\diagrama{objetivos-16}

\subsection{(17.1) Reabastecimiento: hora pico}
\diagrama{objetivos-17.1}

\subsection{(17.2) Reabastecimiento: hora pico}
\diagrama{objetivos-17.2}

\subsection{(18.1) Escalabilidad del software}
\diagrama{objetivos-18.1}

\subsection{(18.2) Escalabilidad del software}
\diagrama{objetivos-18.2}

\subsection{(19)   Baja de bicicletas}
\diagrama{objetivos-19}

\subsection{(20)   Localizaciones de bicicletas}
\diagrama{objetivos-20}


%%%%%%%%%%%%%%%%%%%%%%%%%%%%%%%%%%%%%%%%%%%%%%%%%%%%%%%%%%%%%%%%%%%%%%%%%%%%%%%
%% Escenarios hipotéticos                                                    %%
%%%%%%%%%%%%%%%%%%%%%%%%%%%%%%%%%%%%%%%%%%%%%%%%%%%%%%%%%%%%%%%%%%%%%%%%%%%%%%%


\section{Escenarios hipotéticos}


Los siguientes escenarios ilustran de manera informal situaciones representativas del funcionamiento esperado del sistema.


\subsection{Escenario 1: Registro de usuario}

Un habitante de Mar Chiquita desea comenzar a utilizar las bicicletas del sistema como su medio de transporte. Para conseguirlo ingresa a la página web en donde se encuentra el formulario de registro de usuarios, y procede a completarlo con su nombre completo, su número de DNI, su dirección de correo electrónico y una contraseña para proteger el acceso a sus datos de uso del sistema. Al envíar los datos solicitados, se le confirma que la registración fue exitosa. A partir de ese momento, este habitante ya es considerado un usuario del servicio.


\subsection{Escenario 2: Usuario retira bicicleta, y más tarde la devuelve}

Un usuario registrado se dirige a una estación de la red de ciclovías y expresa su intención de retirar una bicicleta al empleado de la estación. Éste le solicita el DNI al usuario con el fin de verificar su identidad con el mismo y también para poder cargar correctamente los datos del retiro de la bicicleta en el sistema. Luego de la verificación, el empleado devuelve el DNI, y si el usuario no se encuentra penalizado, permite al usuario a tomar una bicicleta. A continuación ingresa la solicitud en el sistema utilizando el DNI del usuario y el ID de la bicicleta; el sistema registra el retiro con fecha y hora del momento.

Luego el usuario ya puede comenzar a utilizar la bicicleta retirada. Se traslada hasta la próxima estación en su recorrido en menos de una hora, y expresa allí su intención de devolver la bicicleta. El empleado de esta otra estación le solicita el DNI al usuario para verificar su identidad y poder registrar la devolución en el sistema. Luego de la verificación, carga la devolución en la interfaz del software indicando número de DNI del usuario, ID de la bicicleta y estado de la misma (sana/rota) al momento de la devolución. El sistema registra la devolución con fecha y hora, y el usuario se retira.


\subsection{Escenario 3: Usuario intenta realizar el retiro/devolución en una estación offline}

Un usuario se dirige a una estación para poder retirar/devolver una bicicleta, pero la estación se encuentra sin servicio de internet.

Luego de realizar la verificación de la identidad del usuario mediante su DNI, al momento de registrar la operación en el sistema de software, en lugar de eso el empleado de la estación se comunica por handy con otra estación que sí posee conectividad a internet, y delega el registro de la operación a su compañero de la segunda estación.

Si el usuario pretendía devolver una bicicleta, la misma es aceptada por el empleado de la estación. Y en el caso de que haya querido retirar una bicicleta, si hay stock, se le ofrece una de las disponibles en la estación.


\subsection{Escenario 4: Usuario devuelve bicicleta después de una hora de retirarla}

Un usuario se dirige a una estación de bicicletas para devolver la bicicleta que tiene en su poder, pero transcurrió más de una hora desde que la retiró (el tiempo máximo permitido de retención de una bicicleta).

El proceso de devolución transcurre por su curso normal, pero una vez que el empleado de la estación carga la devolución en la interfaz del software, el sistema marca al usuario como penalizado y le envía un mail notificándole el hecho. Mientras el usuario esté penalizado, no podrá retirar bicicletas. Asímismo se le ofrece pagar una multa para normalizar su situación.

Luego de recibir la bicicleta del usuario, éste se retira de la estación.


\end{document}